
\documentclass[11pt]{article}
\usepackage{amsmath,amsthm,amssymb,graphicx}

\title{Facial Detection and Recognition on RaspberryPi: Security can be Cheap and Smart}
\author{Foley, Patricia\\Volkmann, Landon \\Yadav, Hari\\Zakari, Abdulmuhaymin}
\date{\today}

\begin{document}

\makeatletter
    \begin{titlepage}
        \begin{center}
            \includegraphics[width=0.7\linewidth]{UMKC.jpg}\\[4ex]
            {\huge \bfseries  \@title }\\[2ex] 
            {\LARGE  \@author}\\[30ex] 
            {\large \@date}
        \end{center}
    \end{titlepage}
\makeatother
\thispagestyle{empty}
\newpage

%Add content for page two here (useful for two-sided printing)
\thispagestyle{empty}
\newpage

\maketitle
\setcounter{page}{1} %Start the actually document on page 1

\begin{abstract}
The development of this project will be vertical in nature to demonstrate proof of concept rather than extensibility. In short, the Group 3 project team has developed a smart security system on the RaspberryPi at minimal cost. 
The system will use a motion sensor and camera to gather data, a convolutional neural network for facial detection and recognition, and finally a push notification based output. The targeted use case of this technology is a household security system. A homeowner will point the system at the door, and the system will notify the homeowner via push notification that they have a guest as well as the system’s best guess at the guest’s name. For this particular implementation, facial data will be manually created for demonstration purposes. Future implementations of the system may include but are not limited to: Facebook profile data, LinkedIn profile data, Google Photos image data, etc.
The end goals of this project may be analyzed on a few different planes. In the meta-educational plane, this project will serve to demonstrate Group 3’s competence in Machine Learning techniques, basic circuitry, and comfort with the Internet of Things. In terms of practical application, this project may serve to demonstrate the potential effectivity of machine learning with simple and cost effective components like the RaspberryPi.  

\end{abstract}
%
%\section{Introduction}
%...
%
\end{document}